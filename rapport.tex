\documentclass[a4paper,12pt]{report}

\usepackage[utf8]{inputenc}
\usepackage[frenchb]{babel}
\usepackage[T1]{fontenc}
\usepackage{lmodern}
\usepackage{layout}
\usepackage[top=2cm, bottom=2cm, left=3cm, right=2cm]{geometry}
\usepackage{setspace}
\usepackage{verbatim}
\usepackage{moreverb}
\usepackage{listings}
\usepackage{graphicx}
\usepackage{shorttoc}
\usepackage[nonumberlist]{glossaries}
\usepackage{xcolor}

% Têtes de chapitre
\usepackage{xcolor}
%\usepackage{biblatex} 

% Tête de chapitre
\makeatletter
\newlength{\chapter@number@width}
\def\@makechapterhead#1{%
  {\normalfont
  \setlength{\parindent}{0pt}%
  \vspace*{10pt}%
  \settowidth{\chapter@number@width}{%
    \hbox{\color{white}\LARGE\bfseries
          \hspace{\dimexpr 1mm+3pt}%
          \thechapter
          \hspace{\dimexpr 1mm+3pt}%
    }}
  \hbox{%
    \vtop{%
      \hsize=\dimexpr\chapter@number@width+\tabcolsep+2\fboxrule+\tabcolsep
      \begin{tabular}[t]{@{}c}
        \scshape\strut\makebox[0pt]{\hspace{0pt plus 1 fill minus 1 fill}\@chapapp\hspace{0pt plus 1 fill minus 1 fill}} \\
        \fboxsep=0pt
        \colorbox{black}{\vbox{%
           \hbox{\vbox to \dimexpr 1mm+3pt{}}
           \hbox{\color{white}\LARGE\bfseries
                 \hspace{\dimexpr 1mm+3pt}%
                 \thechapter
                 \hspace{\dimexpr 1mm+3pt}%
                }
           \hrule height 0.4pt depth 0pt width 0pt
           \hbox{\vbox to 6pt{}}
           \hbox{\parbox{0pt}{\Huge\bfseries\vphantom{E}}}
           }}%
      \end{tabular}%
      }%
    \vtop{%
      \advance\hsize by -\dimexpr\chapter@number@width+2\fboxrule+\tabcolsep
      \hspace*{-0.5cm}\begin{tabular}[t]{c}
        \scshape\strut\vphantom{\@chapapp} \\
        \fboxsep=0pt
        \colorbox{white}{\vbox{%
           \hbox{\vbox to \dimexpr 1mm+3pt{}}
           \hbox{\LARGE\bfseries
                 \hspace{\dimexpr 1mm+3pt}%
                 \phantom{\thechapter}
                 \hspace{\dimexpr 1mm+3pt}%
                }
           \hrule height 0.4pt depth 0pt width \hsize
           \hbox{\vbox to 6pt{}}
           \hbox{\hspace*{20pt}\parbox{\dimexpr\textwidth-2mm-6pt-\chapter@number@width-\tabcolsep-2\fboxrule-20pt}{\Huge\bfseries #1}}
           }}%
      \end{tabular}%
      }%
    }%
  \vspace{50pt}%
  }
}
\def\@makeschapterhead#1{%
  {\normalfont
  \setlength{\parindent}{0pt}%
  \vspace*{10pt}%
  \settowidth{\chapter@number@width}{%
    \hbox{\color{white}\LARGE\bfseries
          \hspace{\dimexpr 1mm+3pt}%
          \thechapter
          \hspace{\dimexpr 1mm+3pt}%
    }}
  \hbox{%
    \vtop{%
      \hsize=\dimexpr\chapter@number@width+\tabcolsep+2\fboxrule+\tabcolsep
      \begin{tabular}[t]{@{}c}
        \scshape\strut\makebox[0pt]{\hspace{0pt plus 1 fill minus 1 fill}\phantom{\@chapapp}\hspace{0pt plus 1 fill minus 1 fill}} \\
        \fboxsep=0pt
        \colorbox{black}{\vbox{%
           \hbox{\vbox to \dimexpr 1mm+3pt{}}
           \hbox{\color{white}\LARGE\bfseries
                 \hspace{\dimexpr 1mm+3pt}%
                 \phantom{\thechapter}%
                 \hspace{\dimexpr 1mm+3pt}%
                }
           \hrule height 0.4pt depth 0pt width 0pt
           \hbox{\vbox to 6pt{}}
           \hbox{\parbox{0pt}{\Huge\bfseries\vphantom{E}}}
           }}%
      \end{tabular}%
      }%
    \vtop{%
      \advance\hsize by -\dimexpr\chapter@number@width+2\fboxrule+\tabcolsep
      \hspace*{-0.5cm}\begin{tabular}[t]{c}
        \scshape\strut\vphantom{\@chapapp} \\
        \fboxsep=0pt
        \colorbox{white}{\vbox{%
           \hbox{\vbox to \dimexpr 1mm+3pt{}}
           \hbox{\LARGE\bfseries
                 \hspace{\dimexpr 1mm+3pt}%
                 \phantom{\thechapter}
                 \hspace{\dimexpr 1mm+3pt}%
                }
           \hrule height 0.4pt depth 0pt width \hsize
           \hbox{\vbox to 6pt{}}
           \hbox{\hspace*{20pt}\parbox{\dimexpr\textwidth-2mm-6pt-\chapter@number@width-\tabcolsep-2\fboxrule-20pt}{\Huge\bfseries #1}}
           }}%
      \end{tabular}%
      }%
    }%
  \vspace{50pt}%
  }
}
\makeatother

% Verbatim langage
%\lstset{
%	language=Python,
%	basicstyle=\footnotesize,
%	basicstyle=\ttfamily\small,
%	keywordstyle=\color{blue}\bfseries\emph,
%	stringstyle=\color{black!60},
%	columns=flexible,
%	tabsize=4,
%	extendedchars=true,
%	showspaces=false,
%	showstringspaces=false,
%	breaklines=true,
%	breakautoindent=true,
%	captionpos=b,
%	numbers=left,
%	numberstyle=\footnotesize,
%	stepnumber=1,
%	numbersep=10pt,
%	frame=single
%}


% Redéfinition de commandes
\renewcommand{\baselinestretch}{1.5}
\renewcommand\thesection{\arabic{section}}
\renewcommand\thesection{\arabic{chapter}.\arabic{section}}

% Définition de commandes
%\newcommand{\EX}{\emph{Exemple}}

%Mots sans césures
%\hyphenation{Exemple} 

% Désactiver les alinéas automatiques
\parindent = 0pt

% Elements pour la page de garde
\title{Titre}
\author{Philippe Chauvelin}
\date{\today}

% Création du glossaire
\makeglossaries
%\newglossaryentry{erp}{
%	name={ERP },
%	text={ERP*},
%	description={\textbf{(Enterprise Resource Planning) :} Progiciel permettant de gérer l'ensemble des processus d'une entreprise en intégrant l'ensemble de 
%	ses fonctions comme la gestion des ressources humaines, la gestion financière et comptable, l'aide a la décision, la vente, la distribution 
%	l'approvisionnement, la production ou encore du e-commerce. Parfois appelé \textbf{PGI (Progiciel de Gestion Intégré)} avec la dénomination française}
%}
%
%\newglossaryentry{opensource}{
%	name={OPEN SOURCE :},
%	text={Open Source*},
%	description={Logiciels dont la licence respecte les critères définis par l'association Open Source Initiative (OSI). Parmi ces critères, on retrouve 
%	la possibilité de libre redistribution, d'accès au code source et de Travaux dérivés}
%}

% Début du document	
\begin{document}
	% Page de garde 
	\begin{titlepage}
		\begin{center}
			Philippe CHAUVELIN - François-Guillaume Ribreau\\
			CSII 3\ieme~année\\
		\end{center}
		\hrulefill
		\vspace{7cm}
		\begin{center} 
			\LARGE \textbf{Les attaques XSS}\\
			\vspace{6cm}
		
			\begin{tabular}{cp{4cm}c}
				\includegraphics[height=75px]{images/logo_epsi.jpg}\\
				\textbf{E.P.S.I - NANTES}\\
				114 rue des Hauts Pavés\\
				BP 41903\\
				44019 NANTES CEDEX 01\\
			\end{tabular}
		\end{center}
	\end{titlepage}
	\newpage
	
	% Page vierge
	\newpage
	\null
	% Met le compteur de numérotation à zéro et n'affiche pas le numéro de page remerciement
	\thispagestyle{empty}
	\setcounter{page}{0}
	\newpage
	
	
	% Sommaire
	\shorttableofcontents{Sommaire}{0}
	\setcounter{page}{1}
	\thispagestyle{empty}
	\newpage
	
	% Introduction
	\chapter*{Introduction} % (fold)
		% Ajout de Introduction dans le sommaire mais sans numéro		
		\setcounter{section}{1}
		
		\addcontentsline{toc}{chapter}{Introduction}
		
		Contenu Introduction
		
			
		% chapter Introduction (end)
		\newpage
	
	% Chapitre 1
	\chapter{Chapitre 1} % (fold)
	% Section 1	
	\section{Section 1}	
		Contenu Section 1
		
		% Section 1 (end)
		\newpage
	
	% Section 2
	\section{Section 2}
		Contenu Section 2
		
		% Section 2 (end)
		\newpage
	% Chapitre 1 (end)
	
	% Chapitre 2
	\chapter{Chapitre 2} % (fold)
		Contenu Chapitre 2
		
		% Chapitre 2 (end)
		\newpage
	
	% Chapitre 3
	\chapter{Chapitre 3} % (fold)
		Contenu Chapitre 3
	
		% Chapitre 3 (end)
		\newpage
	
	% Conclusion
	\chapter*{Conclusion} % (fold)
	\addcontentsline{toc}{chapter}{Conclusion}
		Contenu Conclusion

		% Conclusion (end)
		\newpage
	
	% Glossaire
	\renewcommand\glossaryname{Glossaire}
	\addcontentsline{toc}{chapter}{Glossaire}
	\printglossaries
	\newpage
	
	% Bibliographie
	\addcontentsline{toc}{chapter}{Bibliographie}
	\bibliographystyle{plain} % Le style est mis entre crochets. 
	\nocite{*}		
	\bibliography{bibli} % mon fichier de base de données s'appelle bibli.bib
	\newpage
	
	% Liste des figures
	\addcontentsline{toc}{chapter}{Liste des figures}
	\listoffigures
	\newpage
	
	% Annexes	
	\chapter*{Annexes} % (fold)
	\addcontentsline{toc}{chapter}{Annexes}
	
%	\section*{Annexe 1 - Mindmap récapitulative des modules utiles en fonction des besoins}
%	\addcontentsline{toc}{section}{Annexe 1 - Mindmap récapitulative des modules utiles en fonction des besoins}
%		\begin{figure}[!b]
%			\begin{center}		
%				\includegraphics[width=15.5cm]{images/openerp_mindmap.jpg}
%			\end{center}
%		\end{figure}
%	\newpage
	
	% Table des matières
	\addcontentsline{toc}{chapter}{Tables des matières}
	\tableofcontents
	\newpage
\end{document}